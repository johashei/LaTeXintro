% En LaTeXfil slutter med .tex
% Kommentarer begynner med %.

% Delen før \begin{document} heter "preamble". Her setter du innstillinger som
% gjelder for hele dokumentet og angir hvilke pakker du bruker.
% Pakker importeres med \usepackage{}. Hver pakke har sin egen dokumentasjon som
% du kan søke opp på https://www.ctan.org

\documentclass[a4paper, 12pt, norsk]{article}
% En side har størrelse A4, normal tekststørrelse er 12pt, språk er bokmål (se
% \usepackage{babel} under). ‘article’ er bra for de fleste korte dokumenter,
% som f.eks. oppgavebesvarelser.
% Se https://en.wikibooks.org/wiki/LaTeX/Document_Structure#Preamble for detaljer
% om innstillingene.

\usepackage[utf8]{inputenc}
% Bruk UTF8 enkoding. Du kan lese mer om tekstenkoding og hvorfor det er viktig her:
% https://www.w3.org/International/questions/qa-what-is-encoding

\usepackage[norsk]{babel}
% Bruk ‘norsk’ for å skrive bokmål, ‘nynorsk’ for å skrive nynorsk.

\usepackage{physics} % enklere kommandoer for å skrive bl.a. vektorer, deriverte,
% matriser og notasjon ofte brukt i fysikk.

\usepackage{siunitx} % kommandoer for SI-enheter.

\usepackage{listings}
% Denne pakken lar deg inkludere kode i dokumentet.

\usepackage{cleveref} % smartere henvisninger enn den innebygde kommandoen \ref.

% Innstillinger for teksten, hvis du vil ha noe annet enn standardinstillingene:
\setlength{\parindent}{1em} % innrykk på begynnelsen av avsnitt. 1em er bredden til M.
\setlength{\parskip}{1ex} % ekstra avstand mellom avsnitt. 1ex er høyden til x.
\linespread{1} % linjeavstand

% Endre hvordan seksjonsnummer skrives:
\renewcommand{\thesection}{\Roman{section}}
\renewcommand{\thesubsection}{\Roman{section}--\arabic{subsection}}
\renewcommand{\thesubsubsection}{\alph{subsection})}


\title{Introduksjon til \LaTeX} % tittel
\author{Johannes Sørby Heines} % forfatter

% Dette er slutten på preamble. Hvis du vil bruke samme preamble i flere dokumenter
% kan du lagre den i en egen min_preamble.tex fil, og så skrive \input{min_preamble.tex}
% før \begin{document}. Kommandoen \input{} gjør at LaTeX tolker innholdet i inputfilen 
% som om du hadde skrevet det rett inn i hovedfilen.

\begin{document}

\maketitle  % Setter opp tittel, forfatter og dato.

\begin{abstract} % Sammendrag som dette er nødvendig for vitenskapelige rapporter og artikler.
	Dette dokumentet er laget for å gi en innføring i \LaTeX, med nok informasjon til å skrive
	det som forventes av oppgavebesvarelser og rapporter i løpet av bachelorstudiet.
	Jeg forsøker å gi en forståelse av prinsippene i \LaTeX som vil gjøre det lettere å lete
	seg fram til mer informasjon der det trengs.
	Forslag til forbedringer mottas med takk på \href{github}. Der ligger også den nyeste
	versjonen av dette dokumentet.
\end{abstract}

\section{Dette er en overskrift.}
Hvis du bruker \lstinline[language=tex, basicstyle=\ttfamily]$\documentclass[...]{report}$
har du også kapitler, som kommer før overskrifter.

\subsection{Dette er en underoverskrift.}
Teksten ser lik ut uavhengig av hvor mange overskriftnivåer du har.

\subsubsection{Dette er en underunderoverskrift.}
Hvis du trenger tre overskriftnivåer bør du tenke deg om. Blir teksten klarere
om den omstruktureres?
brukes ikke innrykk.

Som du ser over kommer det ikke innrykk på første avsnitt etter en overskrift, innrykket
kommer derimot når et avsnitt følger et annet.

\section{Nummerering}

\lstset{language=tex, basicstyle=\ttfamily} % Setter listings instillinger for av vise LaTeX-kode.

\subsection{Overskrifter}

Du kan fritt endre hvordan overskrifter nummereres ved å redefinere kommandoene
\lstinline$\thesection$, \lstinline$\thesubsection$, og \lstinline$\thesubsubsection$.
Prøv deg fram og finn hva som passer best.

\subsection*{Overskrift uten nummer}
Du kan også lage en overskrift uten nummer ved å sette en * mellom kommandoen og \{\}, slik:
\lstinline$\subsection*{Overskrift uten nummer}$.
Nest overskrift vil fortsette fra den forrige nummererte overslriften, som dette:

\subsection{Henvisninger}\label{ov:henvisninger}
Det kan ofte være hensiktsmessig å vise til en bestemt seksjon i teksten. Hvis du skriver
f.eks. ``Se avsnitt II--2'', men så bestemmer deg for å flytte denne delen eller legge til
en annen del før, må du endre alle henvisninger manuelt.
For å slippe det kan du sette merker på overskrifter med kommandoen \lstinline$\label{}$
 og henvise til den med \lstinline$\ref{}$ eller \lstinline$\cref{}$. Den første er
innebygd i \LaTeX og gjør ikke annet enn å gi deg nummeret: \ref{ov:henvisninger}.
Den andre ligger i pakken cleveref og vet også hva du refererer til: \cref{ov:henvisninger}.


\section{Matematikk}

En av hovedgrunnene til å bruke \LaTeX~er å skrive matematikk. Det finner flere måter å
gjøre dette på, som passer til forskjellige formål.

Når matematikken skal være integrert i teksten -- såkalt ``inline'' -- skrives den mellom
\lstinline{\( \)}. For eksempel \(r\exp(\theta i) = r\cos(\theta) + ir\sin(\theta)\). For
uttrykk som krever litt mer plass brukes \lstinline{\[ \]} -- kalt ``display'', for eksempel
\[
	f(x) \approx \sum_{n=0}^{\infty} \frac{(x-a)^n}{n!}\dv[n]{f}{x}\qty(a) .
\]
Husk at også likninger som skrives på denne måten skal kunne leses som en del av teksten, så
sett komma eller punktum etter dem.


Ren \TeX, som \LaTeX~er basert på, bruker \lstinline{$ $} og \lstinline{$$ $$} for henholdsvis
inline- og display-matematikk, og dette fungerer også i \LaTeX, men kommandoene over er anbefalt.

For lengre utledninger er det fint å kunne sette likninger pent under hverandre, det gjør vi
med \lstinline$align$. Her brukes \lstinline{&} for å markere referansepunktet, og \lstinline{\\}
for å markere linjeskift.

\begin{align}
	\dv{N}{t} &= -\frac{N}{\tau}  \label{eq:difflikning} \\
	\frac{1}{N}\dv{N}{t} &= -\frac{1}{\tau}  \nonumber \\
	\int \frac{1}{N} \dd{N} &= \int -\frac{1}{\tau} \dd{t} \\
	\ln(N) &= -\frac{t}{\tau}  \nonumber \\
	N &= e^{-t/\tau} \label{eq:svar}
\end{align}

Som med overskrifter kan du bruke \lstinline$\label{}$ og \lstinline$\ref{}$ eller 
\lstinline$\cref{}$ for å enkelt vise til en bestemt likning.
For eksempel er \cref{eq:svar} formelen for radioaktivt henfall.
\lstinline$\nonumber$ fjerner nummereringen for én likning, hvis du vil fjerne nummerering 
for alle kan du bruke \lstinline$align*$.



\end{document}
